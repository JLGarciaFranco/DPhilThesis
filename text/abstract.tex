This thesis investigates the representation of the variability and teleconnections of the American Monsoon System in the Met Office Hadley Centre general circulation models. 
The models exhibit a number of important biases including an overestimation of the strength of the East Pacific Inter-tropical Convergence Zone (ITCZ) and the position of the Atlantic ITCZ is simulated southward of the observed position. 
The representation of the seasonality and magnitude of monsoon precipitation in the North American and Central American monsoons has improved compared to previous generations of the models, however, the spatial distribution of precipitation in the South American Monsoon is still poorly represented due to biases in the equatorial Atlantic sea-surface temperatures.

These models stand-out for their representation of the seasonal cycle of precipitation in southern Mexico and Central America, where a bimodal signal known as the Mid-summer drought (MSD) has been observed for centuries. 
A wavelet transform method is developed to diagnose the timings of the bimodal seasonal cycle and results illustrate that the method can be used to diagnose monsoon timings in any monsoon region for any precipitation time-series, including climate model output.
Using this method, a number of theories that explain the existence and timing of the MSD signal in southern Mexico and Central America are evaluated using reanalysis and CMIP6 simulations.
These result show that the most consistent theory explains the MSD as a result of seasonally varying moisture transport driven by the seasonal cycle of the moisture transport in the Caribbean Sea. 

The simulations show that the teleconnections of El Niño-Southern Oscillation to the American Monsoon System associated with the perturbation to the Walker circulation are linked with the stratospheric quasi-biennial oscillation (QBO). 
An analysis of the CMIP6 pre-industrial experiments of HadGEM3 and UKESM1 shows that the tropical route of QBO teleconnections in the simulations involves ocean-atmosphere phenomena such as the ITCZs, the Indian Ocean Dipole and the Walker circulation. 

Atmosphere-only and coupled ocean-atmosphere experiments are performed where the zonal winds in the equatorial stratosphere are specified to improve the representation of the QBO and evaluate directly the impact of the QBO on tropical climate. 
The experiments with the relaxation of the winds result in weaker or null response to the QBO winds, in contrast to the free-running models. 
The results of these experiments highlight the importance of SST feedbacks and the interaction of tropical waves with the zonal flow in the stratosphere to determine the interaction of the QBO on tropical convection.