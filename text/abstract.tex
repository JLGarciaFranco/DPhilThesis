This thesis investigates the representation of the variability and teleconnections of the American Monsoon System in the state-of-the-art general circulation models. 
The models exhibit several biases including an overestimation of the strength of the East Pacific Inter-tropical Convergence Zone (ITCZ) and the position of the Atlantic ITCZ.
The representation of the seasonality and magnitude of monsoon precipitation in the North American and Central American monsoons has improved compared to previous generations of the models, however, the spatial distribution of precipitation in the South American Monsoon is still poorly represented due to biases in the equatorial Atlantic sea-surface temperatures (SSTs)s.
\vspace{1mm}
These simulations reasonably represent the seasonal cycle of precipitation in southern Mexico and Central America, where a bimodal signal known as the Mid-summer drought (MSD) has been important for agriculture in the region for centuries. 
A wavelet transform method is developed to diagnose the timings of the bimodal seasonal cycle and results illustrate that the method can diagnose monsoon timings in any monsoon region for any precipitation time series, including climate model output.
Using this method, several theories that explain the existence and timing of the MSD signal in southern Mexico and Central America are evaluated using reanalysis and climate models.
These results suggest that the MSD can be most consistently explained through the effect of seasonally varying moisture transport driven by the low-level flow in the Caribbean Sea.
\vspace{1mm}
The influence of the quasi-biennial oscillation (QBO) teleconnections is explored in several climate model simulations, and the results suggest that the QBO could affect ocean-atmosphere phenomena. 
% such as the ITCZ, the Indian Ocean Dipole (IOD) and the Walker circulation. 
More frequent positive phases of El Niño-Southern Oscillation and the Indian Ocean Dipole, and a weaker Walker circulation are found during QBOW compared to QBOE in the models. 
\vspace{1mm}
Atmosphere-only and coupled ocean-atmosphere experiments are performed where the zonal winds in the equatorial stratosphere are specified, by relaxing them towards reanalysis data. 
The representation of the QBO is improved in these experiments, particularly in the lower stratosphere. However, the surface impacts of the QBO in the experiments with the relaxation result in a weaker tropical response to the QBO phase compared to the free-running models. These results imply that the relaxation has disrupted processes and/or feedbacks that are important for the surface impact of the QBO. 
The results of these experiments highlight the importance of SST feedbacks and the interaction of tropical waves with the zonal flow in the stratosphere for the interaction of the QBO on tropical convection.