%\begin{savequote}[8cm]
%\textlatin{Neque porro quisquam est qui dolorem ipsum quia dolor sit amet, consectetur, adipisci velit...}

%There is no one who loves pain itself, who seeks after it and wants to have it, simply because it is pain...
%  \qauthor{--- Cicero's \textit{de Finibus Bonorum et Malorum}}
%\end{savequote}

\chapter{\label{ch:8-conclusions}Conclusions} 

%\minitoc
This chapter summarises the main findings and conclusions of this thesis, discusses the limitations of this research and potential future work.

\section{Summary}

\paragraph{Biases in the dynamics of the American monsoon:}
Assessment of GCM biases in CMIP cohorts is seldom done with emphasis on biases relevant for the monsoons in the Americas. 
%of the AMS is seldom assessed in CMIP models. 
For this reason, less is known about the relationship between large and regional-scale biases for the AMS region compared to other monsoons. 
The first results chapter aims to provide a detailed account of the biases relevant for American monsoons in the CMIP6 MOHC simulations of UKESM1 and HadGEM3. 
The choice of simulations allowed to compare the effect of historical forcing, horizontal resolution, SST biases and the treatment of Earth System processes for the representation of the dynamics of the monsoon.
A key bias diagnosed in Chapter \ref{ch:4-ams} is the overstimation of precipitation in the boreal summer ITCZ in the East Pacific Ocean. 
Similarly, the position of the austral summer Atlantic ITCZ is biased southward in the coupled experiments, leading to biases in the dynamics of the South American monsoon. 
This bias is reduced in the atmosphere-only experiments suggesting a key role played by equatorial Atlantic SST biases for the representation of precipitation over land and in the ITCZ region.
The dynamics and seasonality of the North American monsoon and MSD regions are relatively better represented than for the South American monsoon.
The Earth System processes represented by UKESM1 do not considerably improve the representation of the monsoon compared to HadGEM3. 
However, increasing the horizontal resolution improves a number of biases in the large-scale dynamics and monsoon rainfall. 
This chapter finds that the MOHC models are able to simulate a bimodal signal in the seasonal cycle of precipitation, known as the Mid-summer drought, as well as relationships between the stratospheric QBO and ENSO teleconnections in the tropics. 

\paragraph{A portable method to diagnose monsoon timings:}
Existing methods used to separate the timings in the seasonal cycle in a monsoon are usually tailored to a dataset, which complicates comparisons of multiple datasets or simulations in which the seasonal cycle or climatological precipitation is non-stationary. 
A wavelet transform method is developed using the Haar wavelet aiming to diagnose onset and retreat as sharp or abrupt changes to the signal of precipitation.
The results obtained using the WT method are comparable to other methodologies and the portability of the method is illustrated using observational datasets, reanalysis and climate model output. 
Results show that the WT method reasonably capture the mean dates of onset and retreat as well as changes in the meteorological conditions associated with the monsoons in India and North America compared to existing methodologies. 
Furthermore, the method can also separate the timings of a bimodal signal by finding the dates where the drier period begins and ends. 
The diagnosis of the timings of the MSD simulated by the MOHC simulations using this methodology show similar results to observations, indicating a good representation of the intricate seasonal cycle of precipitation in the MSD region by these models.


\paragraph{A limited role for SST-cloud-radiative feedbacks in the Mid-summer drought precipitation:}
Several theories exist that explain the ocurrence of a bimodal signal of precipitation in Central America, southern Mexico and the Caribbean. 
The most cited hypothesis suggests that a feedback mechanism between the East Pacific SSTs and cloud-radiative processes is the cause for the two-peaks and trough in the seasonal cycle of precipitation. 
A relatively more recent study suggests that the amount of solar radiation reaching the surface, driven by the solar declination angle crossing these latitudes twice, is the leading mechanism. 
Chapter \ref{ch:6-msd} uses the WT method to separate the timings of the seasonal cycle and investigate more closely these two hypotheses in the CMIP6 MOHC simulations and in ERA5.
No evidence is found that the East Pacific SST variability drives the precipitation, as expected by the first mechanism, or that the absorbed solar radiation is the driving mechanism, as suggested in the second hypothesis.
 Rather, the results suggest that absorbed solar radiation by the surface covaries with precipitation and cloudinness, and that East Pacific SSTs do not follow the predictions of the first theory.
The conclusion of this analysis is that the observed interannual varability or the differences in the representation of the MSD in the simulations cannot be explained through the arguments of these two mechanisms, suggesting other factors may be more important for the seasonality of rainfall in the region. 

\paragraph{The role of moisture transport for precipitation in the MSD region:}
The horizontal moisture transport occuring in the Caribbean is found to explain some aspects of the MSD in the CMIP6 MOHC simulations and in reanalysis. 
The vertically integrated moisture transport and total water content decrease considerably from the period of the first peak of rainfall to the drier MSD period, and similarly increase in late summer during the second wet period in both simulations and ERA5.
The moisture transport and total water content changes are explained to a certain extent by the seasonality in the CLLJ, i.e., the low-level zonal flow in the Caribbean. 
The CLLJ variability skilfully explains the precipitation increases during the transition from the MSD period to the second peak period. 
However, the magnitude of the precipitation decrease in early summer, from the first peak to drier MSD periods is not as well explained by the CLLJ as it is by the integraed moisture transport. 

\paragraph{The moist static energy budget provides useful insight to the MSD problem:}
The moist static energy (MSE) budget framework is a useful technique to investigate physical processes associated with tropical precipitation. 
The MSE budget is used in Chapter \ref{ch:6-msd} to investigate how the MSE changes in the MSD region in the simulations and in the reanalysis and how the budget terms could be relate to variations in precipitation. 
A strong relationship is found between the vertical advection term of the budget and precipitation, indicative of the relationship between vertical velocity and precipitation in the tropics. %ity at the pentad-scale.
The horizontal advection of MSE also varies notably between the wet and dry periods, however, there was little evidence to indicate a direct link between the horizontal advection term and precipitation changes. 
Both models and reanalysis suggest a limited role for surface fluxes in the variabilty of precipitation, although considerable biases in the surface energy budget of the models are apparent from this analysis. 
In short, the MSE budget framework may be useful for future work, albeit some improvements to the computation of the budget which are described in the following section.

\paragraph{The tropical route of the QBO teleconnections in a GCM:}
An investigation of the tropical route of QBO teleconnections in the MOHC models was done in the last chapter of this thesis. 
In the long pre-industrial control experiments of CMIP6, several responses that had previously been reported in observational works were confirmed, for example, the southward shift of the East Pacific ITCZ during QBOW compared to QBOE. 
The main findings of this analysis were robust shifts in the position of the Atlantic and Pacific ITCZs, as well as wetter conditions in the Caribbean Sea and Indian Oceans during QBOW compared to QBOE. 
Most of these impacts were found to be seasonally varying and the season of strongest influence varied from model to model. 
A previously unknown relationship between the QBO phase and the Indian Ocean, characterized by an IOD-like response, was found to be robust to internal variability of the simulations. 
This response is characterised by a zonal dipole characterised by wetter conditions in the western Indian Ocean and drier conditions in the eastern Indian Ocean are more frequently found during QBOW than during QBOE. 
Similarly, El Niño events are more frequent during QBOW and La Niña events during QBOE in observations and in these simulations.
Finally, changes to the strength of the Walker circulation were also diagnosed to be robust; this response is characterised by a weaker overturning during QBOW than during QBOE. 
The results in this chapter suggest an intricate link between the ITCZs, the IOD, ENSO and the Walker circulation, and the phase of the QBO. 

\paragraph{QBO tropical teleconnections in a model with a nudged stratosphere:}
To test the direction of causality in the relationships found in the pre-industrial control experiments, as well as to alleviate biases with the representation of the QBO within these models, a number of experiments are described in which the zonal wind in the equatorial stratosphere were relaxed towards a reanalysis. 
Atmosphere-only and coupled experiments were performed with and without nudging. 
The nudged experiments show a limited sensitivity of the tropical circulation and convective activity to the phase of the imposed QBO.
Most of the relationships found in the CMIP6 experiments are also observed in the Control experiments which suggests that the nudging has dampened or removed the processes that cause these impacts to occur in the simulations with an internally generated QBO. 
In the nudged experiments, the mean state of the Walker circulation is driven closer to observations by the nudging, alleviating biases in the upper-level branch of the circulation. Yet, the variability in the Walker circulation associated with the QBO of the nudged experiments is weaker and of a different sign than in the control experiments.
In short, the nudged experiments show a stronger temperature variance in the UTLS in response to the QBO shear, yet the tropical teleconnections are not stronger than in the control experiments. 

\section{Limitations and future work}

The investigations presented in this thesis leave a number of open questions that require future work. 
The main limitations of each part of the research in this thesis and a discussion on how future research could address these limitations is provided below. 

\paragraph{Analysis of observational and model trends in monsoon timings:}
Chapter \ref{ch:5-wvt} shows that the WT method can be used in any observational or model data and for any monsoon so that the advantage of the method is portability.
In contrast, one limitation of the method is that the WT cannot be computed without in a real-time scenario, so the method is only useful for post-hoc processing. 
This limitation is due to the way the WT is computed, which requires the availability of data several months past the date of monsoon onset or retreat to compute the WT. 
Nevertheless, due to the advantages of the method, one further application of the method would be to diagnose observed and simulated trends in the onset and retreat dates in the global monsoon.
Several regional studies exist that characterise trends in the seasonality of precipitation but due to the nature of the WT method, this diagnosis could be done across the CMIP6 chort of models, the existing reanalysis datasets and all the gridded precipitation datasets available.  
This research could provide a more comprehensive analysis of if and how the seasonality of the global monsoon is changing and whether these changes are also seen in CMIP6 models.

\paragraph{The diagnosis of cloud-radiative effects and other quantities from reanalysis:}
Chapter \ref{ch:6-msd} uses ERA5 data to diagnose several quantities such as cloud-radiative effects that are known to be biased in reanalyses compared to satellite products or other observational datasets.
For example, surface fluxes over land and ocean in ERA5 are also likely to depart from observations.
Throughout the chapter, ERA5 data is used to compare with the CMIP6 models, yet the reanalysis does not assimilate and rather simulates some of the diagnostics used in the chapter.
ERA5 data is used because the precipitation within the reanalysis is also simulated and ought to be physically consistent with the other diagnostics. 
In this way, ERA5 can only be used as a best-model and not as representing the observed processes associated with the MSD.
Further work could use observational products to validate the results found in the chapter. 
In particular, a characterisation of the seasonality of CRE in observational datasets could provide insight to how cloud-radiative feedbacks vary temporally in the rainy season.  

\paragraph{A full moist-static energy budget:}
An accurate computatiton of the MSE budget is difficult in reanalysis and model data because high temporal resolution of all the terms is required for the budget to close. 
This limitation is typically addressed by computing the budget terms online within a model. 
An experiment using the MOHC models that computes the MSE with a high frequency would better show how the budget terms vary at each grid-point with the evolution of the rainy season. 
Similarly, the full ERA5 data was not used to compute the budget. A daily-mean and a coarser horizontal resolution than available were used to compute all the budget terms due to time limitations, which led to the MSE budget calculation not closing exactly. 
A more detailed calculation could provide a more precise quantification of how the budget variations relate to precipitation. 
Future work could also investigate how the shape of the vertical profile of the vertical velocity varies with the stages of the MSD in the East Pacific and over land. 
The shape of this profile is known to determine the vertical advection of MSE and gives relevant information of the state of the convective profile, and, therefore, evaluating temporal changes to the shape of the vertical velocity could also point to relevant processes that could affect the MSD.

\paragraph{QBO teleconnections in a single model:}
Chapter \ref{ch:7-qbo} diagnoses the response of the tropical circulation and precipitation to the phase of the QBO in the MOHC CMIP6 simulations. 
These simulations are all from the same modelling centre and the models are all based on the UM, i.e., these simulations use the same dynamical core and share many of their parametrisation schemes and large-scale circulation biases. 
The teleconnections diagnosed in that chapter could also be analysed in different models of CMIP6 and differences in the properties of the simulated QBO and surface response amongst the models could point to processes that link convection and the QBO in this cohort of models. 
Similarly, experiments using the same model but varying the type of convective or gravity-wave schemes could also highlight which processes are most important to represent connections in the UTLS region, while also pointing to routes for model improvement. 

\paragraph{Nudging the MOHC UM:} 
The relaxation experiments described in the second part of Chapter \ref{ch:7-qbo} suggest that the nudging protocol reduced the connection between the tropical stratosphere and troposphere. 
Several reasons could explain these results but only future work can definitively provide an answer. 
Firstly, several types of nudging are possible with a GCM, for example, nudging can be done through a relaxation of the zonal-mean field and not at all grid-points. 
This type of relaxation would allow waves simulated in the troposphere to propagate to the stratosphere whereas the nudging implemented in this thesis would constrain the wave propagation from the troposphere. 
Therefore, relaxation experiments in a model that (1) has a good representation of the QBO and (2) has the capability of performing a zonal-mean nudging could be better suited to diagnose the directions of causality of the relationships between the QBO and tropical climate found in this chapter.
\\
Secondly, another limitation of the nudging procotol in Chapter \ref{ch:7-qbo} is that the mean state of the Walker circulation was modified slightly by nudging the zonal wind in the stratosphere.
Therefore, a nudging protocol that replicates the observed QBO variability in the circulation and temperature fields in the stratosphere without modifying the mean-state of the tropical circulation could be better suited to diagnose QBO teleconnections in the tropics.

 




