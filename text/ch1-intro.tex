%\begin{savequote}[8cm]
%\textlatin{Neque porro quisquam est qui dolorem ipsum quia dolor sit amet, consectetur, adipisci velit...}

%There is no one who loves pain itself, who seeks after it and wants to have it, simply because it is pain...
%  \qauthor{--- Cicero's \textit{de Finibus Bonorum et Malorum}}
%\end{savequote}

\chapter{\label{ch:1-intro}Introduction} 

%\minitoc



The American Monsoon System (AMS) is the main source of rainfall for the most part of Latin America and southwestern United States. Changes to the amount and location of rainfall over different temporal scales in this region has important consequences for the livelihood of the people that experience depend on the monsoon. Therefore, understanding the mechanisms that cause short or long-term changes to the amount of rainfall in this monsoon is paramount. 

This problem is accentuated by the relatively smaller literature on this monsoon than in other, more studied monsoons. Key questions about general aspects of the variability remain open, and similarly, aspects of the remote-distant effects of other climatic forcings are also not yet understood.

\section{Motivation}

have been less studied than other monsoon regions, even sometimes not regarded as a proper monsoon. Key questions remain for the climate of the region. 
Assessments of GCMs for their representation of the climate of the AMS are scarce, and thus how much can be better understood about the AMS from these models is also unknown. 

One key tool to understand the causes and mechanims for these changes to the monsoon are general climate models. The use of these models to make progress on understanding the real-life monsoons is limited by the capability of the models to reproduce the observed monsoon.  These models are used to provide climate projections of future climate in the AMS. However, GCMs may also be used to understand physical mechanisms associated with climate variability and teleconnections. This understanding can not only improved predictability of the monsoon on various scales but can also push model development to better represent key processes.

 This thesis focuses on the American Monsoon System and the outstanding questions regarding the climate variablity and teleconnections affecting this monsoon.
 
\section{Contribution and aims}
This thesis uses a number of climate models to better understand key outstanding questions of the variability and teleconnections to the AMS. 

The key aims of this thesis are: 

\begin{enumerate}
\item To provide a detailed description of the biases in the representation of the AMS in a state-of-the-art climate model to understand the role of large-scale biases, resolution and the use of chemistry for regional monsoon representation.
\item To quantify the seasonal variability of the monsoon onset, withdrawal and intra-seasonal changes.
\item To investigate the role of stratospheric-tropospheric coupling in the tropics. 
\end{enumerate}

The remainder of this thesis is structured as follows:

\begin{itemize}
\item Chapter 2 provides a review of the literature on key aspects of the American monsoons. The chapter begins by introducing the concepts of monsoons, their different physical interpretations as a global phenomena and the place of regional monsoons in the global scale. Then, the North and South American monsoons are introduced, followed up by a description of rainfall in Central and the literature on a unique monsoon-like regime of precipitation in this region. El Niño-Southern Oscillation is introduced and their impacts over North and South America are described with key emphasis on questions. Finally, the role of the stratosphere for tropical convection has recently regained attention, so this chapter ends with an up-to-date survey on the literature with specific emphasis on the stratospheric quasi-biennial oscillation. 

\item  Chapter 3 evaluates two state-of-the-art CMIP6 models from the Met Office Hadley Centre for their representation of the AMS. The chapter describes large-scale biases over the tropics to then present local scale biases in the precipitation amount and seasonality in key regions of the AMS. 
The representation of ENSO teleconnections is also explored.
The majority of this work has been published in \textit{Weather and Climate Dynamics} \cite{garciafranco2020}.
\item Chapter 4 provides a method that is able to better characterise the MSD timings and strenghths, as a way of analysing the mechanisms of the MSD in observations and models, analysis that is done in chapter 5. The Quasi-biennial Oscillation is proposed to be responsible for the different ENSO teleconnections shown in chapter 3 and are thus further explored using modelling experiments in chapter 6. 
\end{itemize}