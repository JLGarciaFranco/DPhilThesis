\begin{savequote}[8cm]
\textlatin{Neque porro quisquam est qui dolorem ipsum quia dolor sit amet, consectetur, adipisci velit...}

There is no one who loves pain itself, who seeks after it and wants to have it, simply because it is pain...
  \qauthor{--- Cicero's \textit{de Finibus Bonorum et Malorum}}
\end{savequote}

\chapter{\label{ch:1-intro}Introduction} 

\minitoc



\section{Motivation}

The American Monsoon System (AMS) provides the majority of rainfall for the large regions in Latin America and southwestern United States. 
Climate variability and teleconnections to this monsoon system can impact the population through changes in extreme precipitation, the timings of the monsoon or the overall rainfall during the rainy or the dry seasons causing floods or droughts.  

General circulation models (GCMs) have been used to provide climate projections of future climate in the AMS. However, GCMs may also be used to understand physical mechanisms associated with climate variability and teleconnections. 

 This thesis focuses on the American Monsoon System and the outstanding questions regarding the climate variablity and teleconnections affecting this monsoon.
\section{Contribution}

 Chapter 3 evaluates two state-of-the-art CMIP6 models for their representation of the monsoon system. In general, the models show a good representation of the seasonal cycle as they are able to simulated detail aspects such as the Midsummer drought. ENSO teleconnections in these models appear to be non-linear, as are the observations. Chapter 4 provides a method that is able to better characterise the MSD timings and strenghths, as a way of analysing the mechanisms of the MSD in observations and models, analysis that is done in chapter 5. The Quasi-biennial Oscillation is proposed to be responsible for the different ENSO teleconnections shown in chapter 3 and are thus further explored using modelling experiments in chapter 6. 