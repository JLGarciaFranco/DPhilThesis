%\begin{savequote}[8cm]
%\textlatin{Neque porro quisquam est qui dolorem ipsum quia dolor sit amet, consectetur, adipisci velit...}

%There is no one who loves pain itself, who seeks after it and wants to have it, simply because it is pain...
%  \qauthor{--- Cicero's \textit{de Finibus Bonorum et Malorum}}
%\end{savequote}

\chapter{\label{ch:1-intro}Introduction} 

%\minitoc

The American Monsoon System (AMS) is the main source of rainfall for most of
Latin America and the southwestern United States, which are regions where agricultural
activity is economically crucial and where a vast wealth of ecosystems and biodiversity are
present. Changes to the amount, timings and location of rainfall over different temporal scales
have direct consequences for the livelihood of the people and ecosystems in the regions.
Improving our physical understanding of the mechanisms that cause temporal changes to the
AMS rainfall is crucial to improve our medium-range forecasts and our climate predictions
which could ultimately render key information for risk assessments, climate adaptation and
agricultural strategies. In this context, this thesis aims to tackle outstanding questions in the
American Monsoons in a global climate model with a particular interest in better understanding
the physical mechanisms associated with variability and teleconnections of this monsoon.


\section{Motivation}

Temporal and spatial variability of rainfall is important for society throughout the planet for various reasons, but the timing and strength of rainfall is increasingly relevant in agriculturally active and biodiverse hotspot tropical regions \citep{sultan2005,jain2015}. 
In the AMS, changes to rainfall on inter-annual scales can produce long-lived droughts that are associated with crop loss and forest fire intensification \citep{chen2009,harvey2018}.
% But shorter-term changes can also produce profound effects such as flash floods and intense rainfall periods that cause crop loss \citep{devereux2007,avila2016recent}.
A large body of monsoon research is consequently focused on understanding the physical mechanisms responsible for precipitation variability across temporal and spatial scales \citep{wang2017,gadgil2018}. 

The AMS was recognized as a monsoon only after the 1990s, which is relatively recent, as the definition for a monsoon has evolved from an initial dynamical definition based on a reversal of the prevailing winds to an agronomical definition that recognizes the seasonality of precipitation as the dominant feature of a monsoon \citep{wang2017,gadgil2018}. This  means, however, that our understanding of the AMS is more limited compared to other monsoons given the lower number of studies on the AMS from a monsoon perspective compared to other monsoons such as the Indian monsoons, where monsoon forecasts exist since the 19th century \citep{blanford}.
For this reason, several primary questions about general aspects of the AMS remain open including unknown mechanisms for interannual variability and teleconnections.

Recently, theories for general or global monsoon dynamics \citep{bordoni2008monsoons,biasutti2018global,hill2019,geen2020} have arisen to coherently explain the monsoons through a general physical mechanism. Most of these theories aim to explain a global inter-hemispheric band of convection that is driven by the seasonal cycle of solar insolation. Several characteristics of the North and South American monsoons, however, challenge the basic physical inferences or predictions of these theories, which means that these frameworks cannot readily  be applied to the AMS.% to further understand their variability. 


The lack of understanding of the basic physical mechanisms that drive the seasonal cycle of rainfall in southern Mexico and Central America is one example of the gaps in the literature of the AMS. The so-called Midsummer drought is a robust bimodal feature of the seasonal cycle of precipitation during the wet season that has had implications for agricultural practices in the region since the Mayan Empire (AD 800-900) \citep{jobbova2018ritual}. Despite the importance of region-wide agricultural practices, the physical mechanisms that can explain this seasonal variation of rainfall remain disputed over recent years \citep{karnauskas2013,herrera2015,zermeno2019}. 

Climate research in South America has recently focused to investigate the non-linear responses of precipitation of the AMS to teleconnections from the El Niño-Southern Oscillation (ENSO), which occurs just on the western coast of the continent. ENSO phenomena has been well-known by Peruvian fishermen for centuries and has shaped agricultural practices and caused mass migrations \citep{caramanica2020nino}. Nevertheless, the understanding of the effect of ENSO over South America and the AMS in general is still somewhat limited. For example, two ENSO events that are very similar in the central Pacific can cause teleconnections with different locations and strengths and the reasons behind these varying effects are not well understood. 

% partly because of a lack of long-term reliable data over the Pacific Ocean. 
%the reasons for the observed different responses  to the same long-distance ocean-atmosphere forcing from the Pacific Ocean remain unclear, in spite of the well documented impacts of ENSO which has caused many disastrous floods and droughts across the region.

One key tool to understand the causes for regional changes to monsoon rainfall are general circulation models (GCMs). These models are useful to evaluate the roles of climate features such as orography, vegetation-atmosphere and ocean-atmosphere feedbacks, ENSO and their impacts over many aspects of Earth's climate including monsoons \citep{zhou2016}. However, the use of GCMs to address key questions of the AMS has been scarce and detailed accounts of the biases --differences between the simulated climate of a model and the real world -- are rarely done with explicit emphasis on the AMS. 
In other words, GCMs are rarely evaluated in the AMS, so our understanding is deficient both in the knowledge of the relevant biases in current GCMs for the AMS but also because of a relative scarce use of GCMs to address scientific questions related to the AMS. 


 This thesis focuses on the AMS and the outstanding questions regarding the climate variablity and teleconnections affecting this monsoon.
 This thesis begins (Chapter \ref{ch:4-ams}) by evaluating a state-of-the-art climate model, the UK Met Office Unified Model (UM) in the AMS region and comparing the model with several observational datasets, with assessing the roles of biases in the large-scale circulation that affect regional monsoon rainfall biases.
 This assessment highlights that UM model is fit to investigate two outstanding research questions in the AMS and monsoon literature: the physical mechanisms that control the seasonal cycle of rainfall in southern Mexico and Central America, and the role of the tropical stratosphere for tropical convection and monsoons. 
 
 For the first research question, Chapter \ref{ch:5-wvt} describes a new method to determine monsoon timings, including the timings of bimodal regimes of precipitation in observations and climate model output. This method is then used in Chapter \ref{ch:6-msd} to investigate the physical mechanisms of the seasonal cycle of precipitation in southern Mexico and Central America.
%  In a way, each chapter of this thesis builds on the previous chapters. 
 % For that purpose, the thesis first presents a chapter that describes the biases and possible processes responsible for these biases in a state-of-the-art climate model. This
 
\section{Thesis aims and outline}
The main aim of this thesis is to investigate the physical causes of variability and the mechanisms associated with teleconnections to the AMS. The specific key aims of this thesis are: 

\begin{enumerate}
\item To characterize the large-scale biases in a state-of-the-art GCM that are relevant for the representation of rainfall in the AMS.  
\begin{enumerate}
\item To characterize the main biases in the thermodynamical and dynamical features over the large scale tropical domain and the regional AMS sub-domains.
\item To evaluate the roles of large-scale biases, horizontal resolution and the use of Earth system processes for regional monsoon representation.
\item To assess the representation of the teleconnection associated with the main driver of interannual variability, i.e., ENSO in a GCM with specific emphasis on the causes for non-linearity and non-asymmetry in the teleconnections.
\end{enumerate}
%To provide a detailed description of the biases in the representation of the AMS in a state-of-the-art climate model to understand 
\item Evaluate the seasonal variability of the monsoon onset, withdrawal and intra-seasonal changes in the GCM and compare to observational datasets. 
\item Describe and investigate the physical mechanisms associated with the seasonal cycle of rainfall in Central America and southern Mexico by testing previous hypothesis of physical mechanisms within the model.
\item To investigate the role of stratospheric-tropospheric coupling in the tropics and the role of the tropopause for convection in the AMS and for ENSO teleconnections.
\end{enumerate}

The remainder of this thesis is structured as follows:

\begin{itemize}
\item Chapter \ref{ch:2-litreview} provides a review of the literature on key aspects of the American monsoons. The chapter begins by introducing the concepts of monsoons, their different physical interpretations as a global phenomena and the place of regional monsoons in the global scale. Then, the North and South American monsoons are introduced and detail is given on the applicability of large-scale monsoon theories to these regional monsoons. This section is followed by a literature review of the proposed physical mechanisms that drive the seasonal cycle of rainfall in Central America, southern Mexico and the Caribbean. El Niño-Southern Oscillation  impacts over North and South America and, finally, the chapter summarises the literature on stratospheric-tropospheric coupling in the tropics, discussing possible mechanisms by which the stratospheric quasi-biennial oscillation may be influential for tropical convection. 

\item  Chapter \ref{ch:3-methods} describes the observational datasets used in this thesis, composed of four gridded precipitation datasets and one reanalysis dataset: ERA5. The chapter also described the UK Met Office Hadley Centre Unified Model (UM) and the configurations of the UM used in this thesis and the Coupled Model Intercomparison Project phase 6 (CMIP6). 

\item Chapter \ref{ch:4-ams} evaluates the representation of the AMS  in three configurations of the UM model submitted to CMIP6. The chapter describes large-scale biases over the tropics and regional scale biases in the precipitation amount and seasonality in key regions of the AMS. 
ENSO teleconnections are also evaluated over the AMS examinging the non-linearity of simulated and observed teleconnections and the role of ENSO flavours. This chapter highlights relevant questions that are of interest to the wider AMS community that are tackled in the remaining chapters: first, the skill of the models in reproducing a bimodal signal in the seasonal cycle of rainfall in Central America and southern Mexico and second, a possible modulation of ENSO teleconnections by the stratospheric quasi-biennial oscillation.
The majority of the work in this chapter has been published in \textit{Weather and Climate Dynamics} as \cite{garciafranco2020}.

\item Chapter \ref{ch:5-wvt} details a wavelet covariant transform method used to diagnose changes to the timings of the monsoon by determining the onset and retreat dates from precipitation time-series. The method is extended to be used to determine the timings of bimodal regimes, and even whether or not a bimodal regime exists or not in a given region. The method is illustrated in the North American and Indian monsoons, and for the Mid-summer Drought signals of the Caribbean and southern Mexico. This chapter has been published in the \textit{The International Journal of Climatology} as \cite{garciafranco2021}.
\item Chapter \ref{ch:6-msd} uses the method of the previous to investigate the physical mechanisms that cause the bimodal regime of precipitation in southern Mexico in the UM CMIP6 models. The chapter tests elements of three leading hypothesis for the Mid-summer drought, especifically the chapter examines the roles of the East Pacific sea-surface temperatures, the cloud-radiative effects and shortwave radiation and the Caribbean Low-Level Jet. Furthermore, the chapter uses a moist static energy budget, which provides useful insight as to the causes for the changes to precipitation within the rainy season in these regions.
\item Chapter \ref{ch:7-qbo} investigates the tropical route of QBO teleconnections in the pre-industrial control experiments of HadGEM3 and UKESM1.
The observational evidence that links the QBO with the ITCZ, ENSO and the Walker circulation is examined within these simulations to evaluate whether similar relationships to the observations are found within a state-of-the-art GCM. 
The results from these CMIP6 simulations are very similar to observations suggesting robust relationships between the QBO phase, the ITCZs and the Walker circulation.
\item Chapter \ref{ch:7-nudging} follows up on the topic of QBO teleconnections in the tropics by describing the realization of several experiments using a GCM with a nudged stratosphere. These experiments aim to alleviate biases with the representation of the QBO as well as to test the causal pathway through which the tropical troposphere and stratosphere interact with each other. 
The results of this chapter show that when the relaxation was applied most of the teleconnections observed in the control experiments disappear, indicating either that the mechanisms that modulate stratospheric-tropospheric coupling were obscured by the experimental design or that the QBO variability is not driving these teleconnections.
\end{itemize}